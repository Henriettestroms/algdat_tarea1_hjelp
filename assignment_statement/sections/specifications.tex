En esta sección se describen los pasos a seguir para completar la tarea, la cual consiste en desarrollar código en C++ <<\nameref{subsec:implementations}>> y en realizar un informe <<\nameref{subsec:report}>>. Se espera que cada uno de los pasos se realice de manera ordenada y siguiendo las instrucciones dadas.

\begin{mdframed}
    Abra este documento en algún lector de pdf que permita hipervínculos, ya que en este documento el texto en color \textcolor{blue}{azul} suele indicar un hipervínculo.
\end{mdframed}

\begin{enumerate}[(1)]
    \item 
    En caso de dudas, enviar preguntas \textbf{al foro de la Tarea 1}. 
    
    En caso de cualquier de modificaciones, todas se informarán tanto por aula como por discord.
    
    \item Todo lo necesario para realizar la tarea se encuentra en el repositorio de github que se obtiene mediante la siguiente invitación de GitHub Classroom:
\begin{center}

    \url{https://classroom.github.com/a/0NcKwA6A}
        
\end{center}

\textbf{Nota:} La estructura de archivos y el template del proyecto se encuentran disponibles en el repositorio generado.

\item El repositorio generado contiene toda la información oficial y estructura necesaria para completar la tarea.
\end{enumerate}

\subsection{Implementaciones} \label{subsec:implementations}




\begin{enumerate}[(1)]
    \item \textbf{Ordenamiento y Multiplicación.}
    \begin{itemize}
        \item \textbf{Ordenamiento.}  Implementar en C++ los algoritmos de ordenamiento: \textsc{Insertion Sort}, \textsc{Merge Sort}, \textsc{Quick Sort}, \textsc{Panda Sort} y el algoritmo \textsc{Sort} de la librería estándar de C++. Use la siguiente estructura de archivos:
        
    \begin{itemize}
        \item code/sorting/algorithms/insertionsort.cpp
        \item code/sorting/algorithms/mergesort.cpp
        \item code/sorting/algorithms/quicksort.cpp
        \item code/sorting/algorithms/pandasort.cpp
        \item code/sorting/algorithms/sort.cpp
    \end{itemize}

        \textbf{Importante para Quick Sort:} NO implementar el QuickSort básico. Debe implementar una variación del algoritmo, eligiendo una de las siguientes opciones:
        \begin{itemize}
            \item \textbf{Random Pivot} - Selección aleatoria del pivot
            \item \textbf{Middle Element} - La mediana siempre es el pivot
            \item \textbf{Median-of-Medians} - Algoritmo determinístico O(n) para pivot óptimo
            \item \textbf{Randomized QuickSort} - Con shuffle inicial del arreglo
        \end{itemize}
        
        Puede utilizar implementaciones disponibles en internet, pero debe \textbf{citar debidamente las fuentes} en el código y en el informe. 
    
    \textbf{Nota importante:} El algoritmo \textsc{Panda Sort} es un algoritmo de ordenamiento desarrollado específicamente para esta tarea. La implementación completa se encuentra disponible en el repositorio del curso para su análisis.
    
    El algoritmo \textsc{Sort} ya se encuentra implementado en el repositorio con el fin de que sea incluido en sus mediciones del punto (2) de la subsección <<\nameref{subsec:implementations}>>.
    
        \item \textbf{Multiplicación. } Se deben implementar en C++ los algoritmos de multiplicación \textsc{Naive} y \textsc{Strassen}. Use la siguiente estructura de archivos. 
    \begin{itemize}
        \item code/matrix\_multiplication/algorithms/naive.cpp
        \item code/matrix\_multiplication/algorithms/strassen.cpp
    \end{itemize}
    \end{itemize}

    \item \textbf{Mediciones de tiempo y memoria.}  Implementar el programa que realizará las mediciones de tiempo y memoria en C++ (programas principales) y su respectivo \texttt{makefile}, que ejecutará los algoritmos y generará los archivos de salida en cada uno de los directorios measurements sorting y measurements matrix multiplication con los resultados de cada uno de los algoritmos.
    \begin{itemize}
        \item code/sorting/sorting.cpp
        \item code/matrix\_multiplication/matrix\_multiplication.cpp
    \end{itemize}
    
    \item \textbf{Generación de gráficos. } Implementar el programa que generará los gráficos en PYTHON y que se encargará de leer los archivos generados por los programas principales guardados en measurements sorting y measurements matrix multiplication, para luego graficar los resultados obtenidos y guardarlos en formato PNG en plots sorting y plots matrix multiplication.
    \begin{itemize}
        \item code/sorting/scripts/plot\_generator.py
        \item code/matrix\_multiplication/scripts/plot\_generator.py
    \end{itemize}
    \item \textbf{Documentación.} Documentar cada uno de los pasos anteriores
    \begin{itemize}
        \item Completar el archivo \texttt{README.md} del directorio \texttt{code}
        \item Documentar en cada uno de sus programas, al inicio de cada archivo, \textbf{fuentes de información, referencias y/o bibliografía} utilizada para la implementación de cada uno de los algoritmos.
        \item Para el algoritmo \textsc{Panda Sort}, documentar el diseño y la lógica del algoritmo desarrollado.
    \end{itemize}
\end{enumerate}

\subsection{Informe} \label{subsec:report}
Generar un informe en \LaTeX\ que contenga los resultados obtenidos y una discusión sobre ellos. En el repositorio generado por GitHub Classroom podrá encontrar el Template en el directorio report que \textbf{deben utilizar}, en esta entrega:

\begin{mdframed}

\begin{center}
    
    Template disponible en: \texttt{report/} dentro del repositorio generado
        
\end{center}
\end{mdframed}

\begin{itemize}
    \item No se debe modificar la estructura del informe.
    \item Las indicaciones se encuentran en el archivo \texttt{README.md} del repositorio y en la plantilla de \LaTeX. 
\end{itemize}